% !TEX TS-program = xelatex
%
\usepackage{lmodern}
\usepackage{amsmath}
\usepackage{amssymb}
\usepackage[ngerman]{babel}
\usepackage[procnames]{listings}
\usepackage{listings-rust}
\usepackage{listings}
\usepackage{ulem}
\usepackage{amsthm}
\usepackage{tikz}
\usepackage{wasysym}

\author{}

\lstloadlanguages{Matlab}
\lstloadlanguages{Ruby}

\lstset{%
  basicstyle = \ttfamily\color{black},
  commentstyle = \ttfamily\color{gray},
  keywordstyle = \ttfamily\color{red},
  stringstyle = \color{orange},
  breaklines = true,
  showstringspaces = false,
  procnamekeys={def},
  procnamestyle=\color{green}
}


\begin{document}
  \title{Informationstheorie und Kryptologie: 11. Blatt für 14.6.2018}
  \maketitle

  \section*{28)}

    Bei einer Zahlenmenge ${0, 1, ..., q-2}$, gilt für ein primitives Element $\zeta$:
    \[ \{ 0, 1, ..., q-2 \} \to F \setminus \{ 0 \} , i \to \zeta^i \]
    Für ein primitives Element sind alle $\zeta^i$ innerhalb der Logarithmentafel eindeutig.\\
    Java-Version zur Überprüfung, ob ein Element primitiv ist:\\
    \lstinputlisting[language=Java]{log_table-java/src/log_table.java}
    
    Für $q = 31$ ist das kleinste primitive Element $\zeta = 3$.\\
    Produkte können nach folgender Formeln berechnet werden:
    \begin{equation}
      \begin{split}
        a*b & = \zeta^{(log_\zeta(a)+log_\zeta(b) mod (q-1)} \\
        16*17 & = \zeta^{(log_\zeta(16)+log_\zeta(17) mod (30)} \\
        & = \zeta^{(6+7) mod (30)} \\
        & = \zeta^{13} = 24
      \end{split}
    \end{equation}
    Das Inverse kann nach folgender Formel ermittelt werden:
    \begin{equation}
      \begin{split}
        a^{-1} & = \zeta^{q-1-log_\zeta(a)} \\
        16^{-1} & = \zeta^{30-1-log_\zeta(16)} \\
        & = \zeta^{30-6} \\
        & = \zeta^{24} = 2
      \end{split}
    \end{equation}
        
\end{document}