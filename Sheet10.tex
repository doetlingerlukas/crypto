% !TEX TS-program = xelatex
%
\usepackage{lmodern}
\usepackage{amsmath}
\usepackage{amssymb}
\usepackage[ngerman]{babel}
\usepackage[procnames]{listings}
\usepackage{listings-rust}
\usepackage{listings}
\usepackage{ulem}
\usepackage{amsthm}
\usepackage{tikz}
\usepackage{wasysym}

\author{}

\lstloadlanguages{Matlab}
\lstloadlanguages{Ruby}

\lstset{%
  basicstyle = \ttfamily\color{black},
  commentstyle = \ttfamily\color{gray},
  keywordstyle = \ttfamily\color{red},
  stringstyle = \color{orange},
  breaklines = true,
  showstringspaces = false,
  procnamekeys={def},
  procnamestyle=\color{green}
}


\begin{document}
  \title{Informationstheorie und Kryptologie: 10. Blatt für 7.6.2018}
  \maketitle

  \section*{28)}
  
    Nein, der Klartext ist nicht eindeutig, da durch verschiedenen Schlüsselwerte beim Decodieren auch verschiedene Klartextergebnisse entstehen!\\
    \newline
    Die Eindeutigkeit liegt bei einem Schlüsselwert von 2, da hierbei die einzig mögliche sinnvolle Folge entsteht. Das Ergebnis dabei ist:
    \begin{center}
    	\textbf{S}HE\textbf{S}ELLS\textbf{S}EA\textbf{S}HELLS\textbf{B}Y\textbf{T}HE\textbf{S}EASHORE
    \end{center}
    C++ Version zur ausgabe aller Decodierungsmöglichkeiten:\\
    \lstinputlisting[language=C++]{sheet_10/shift_chiffre.cpp}   
  
\end{document}