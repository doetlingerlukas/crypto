% !TEX TS-program = xelatex
%
\documentclass[11pt,a4paper]{article}

\usepackage{lmodern}
\usepackage{amsmath}
\usepackage{amssymb}
\usepackage[ngerman]{babel}
\usepackage[procnames]{listings}
\usepackage{listings-rust}
\usepackage{listings}
\usepackage{ulem}
\usepackage{amsthm}
\usepackage{tikz}
\usepackage{wasysym}

\author{}

\lstloadlanguages{Matlab}
\lstloadlanguages{Ruby}

\lstset{%
  basicstyle = \ttfamily\color{black},
  commentstyle = \ttfamily\color{gray},
  keywordstyle = \ttfamily\color{red},
  stringstyle = \color{orange},
  breaklines = true,
  showstringspaces = false,
  procnamekeys={def},
  procnamestyle=\color{green}
}


\begin{document}
  \title{Informationstheorie und Kryptologie: 5. Blatt für 19.4.2018}
  \maketitle

  \section*{13)}

  \lstinputlisting[language=Rust]{sheet_5/src/main.rs}

  \section*{14)}

  \begin{align*}
    & H(7/24, 10/24, 7/24) &=\ &2 \cdot H(7/24) + H(10/24) &\approx&\ 1,563202089984486 \\
    & H(6/24, 12/24, 6/24) &=\ &2 \cdot H(6/24) + H(12/24) &=&\       1,5 \\
    & H(8/24, 8/24, 8/24)  &=\ &3 \cdot H(8/24)            &\approx&\ 1,584962500721156
    \\
    \\
    & H(4/8, 0/8, 4/8) &=\ &2 \cdot H(4/8) + H(0/8)        &=&\       2 \cdot H(4/8) + 0 \cdot -\infty = \text{undefiniert} \\
    & H(2/8, 4/8, 2/8) &=\ &2 \cdot H(2/8) + H(4/8)        &=&\       1,5 \\
    & H(3/8, 2/8, 3/8) &=\ &2 \cdot H(3/8) + H(2/8)        &\approx&\ 1,561278124459133
  \end{align*}

  Maximalwert bei drei Argumenten ($r := 3$):

  \begin{align*}
    H(\frac{1}{r}, \frac{1}{r}, \frac{1}{r}) &= ld(r) \\
    H(\frac{1}{3}, \frac{1}{3}, \frac{1}{3}) &= ld(3) \approx 1,584962500721156
  \end{align*}

  Das Maximum ist bei drei Werten eindeutig, da das Maximum bei einem festen $r$ bei $H(\frac{1}{r},\frac{1}{r},\dots,\frac{1}{r})$ liegt.

  \section*{15)}
\end{document}
