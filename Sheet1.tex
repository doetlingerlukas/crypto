% !TEX TS-program = xelatex
%
\documentclass{article}
\usepackage[T1]{fontenc}

\usepackage[utf8]{inputenc}
\usepackage[english]{babel}
\usepackage{upgreek}
\usepackage{amsmath}


\begin{document}
  \title{Sheet 1 - Informationstheorie und Kryptologie}
  \date{}
  \author{}

  \maketitle

\section*{1)}
  \[ \Omega = \{(x,y) |x,y \in \{ 1,2,3,4,5,6 \} \} \]

  \subsection*{a)}
    \[ P \{ Augensumme \leq 2 \} = P(\{ (1,1)\} ) = \frac{1}{36} = 2.78\% \]

  \subsection*{b)}
    \[ P \{ Augensumme \geq 2 \} = P(\{ (4,6), (5,5), (6,4), (5,6), (6,5), (6,6)\} ) = \frac{6}{36} = 16.67\% \]

  \subsection*{c)}
    \[ P \{ Augensumme \in N_{even} \} = \frac{18}{36} = 50\% \]

\section*{2)}
  Anzahl der Möglichkeiten bei einer Ziehung von fünf Karten aus einem Deck mit 52 Karten:

  \begin{align*}
    \binom{52}{5} = \frac{52!}{5!(52-5)!} = 2.598.960
  \end{align*}

  \subsection*{a)}
  Der Drilling kann von dreizehn Werten und drei verschiedenen Farben sein. Das Paar kann eines von den verbliebenen zwölf Werten sein und besteht aus zwei von vier Farben.

  \begin{align*}
    \binom{13}{1}\binom{4}{3}\binom{12}{1}\binom{4}{2} = 3.744
  \end{align*}

  \begin{align*}
    \frac{3.744}{2.598.960} = 0,00144 = 0,144\%
  \end{align*}


  \subsection*{b)}
  Der Flush besteht aus fünf Karten derselben Farbe. Von jeder Farbe gibt es dreizehn Karten. Es gibt vier verschiedene Farben. Von der Zahl ziehen wir die 36 möglichen straight flushes und die vier möglichen royal flushes ab, die jeweils extra gewertet werden.

  \begin{align*}
    \binom{13}{5}\binom{4}{1} - 36 - 4 = 5.108
  \end{align*}

  \begin{align*}
    \frac{5.108}{2.598.960} = 0,00197 = 0,197\%
  \end{align*}

\section*{3)}

  \subsection*{a)}
    \[ P(A \cap B \cap C) = P(A) \cdot P(B) \cdot P(C) \]
    \[ P(\{1, 2, 3, 4\}) \cap P(\{2, 3, 4, 5\}) \cap P(\{3, 4, 5, 6\}) = P(\{3, 4\}) = \frac{2}{6} = 0,\dot{3} \]
    \[ P(\{1, 2, 3, 4\}) \cdot P(\{2, 3, 4, 5\}) \cdot P(\{3, 4, 5, 6\}) = \frac{4}{6} \cdot \frac{4}{6} \cdot \frac{4}{6} = \frac{64}{216} = \frac{8}{27} = 0,2962962963 \]

  \subsection*{b)}
    \[ P(A \setminus B) = P(A) - P(B) \]
    \[ P(\{1, 2, 3, 4\} \setminus \{3, 4, 5, 6\}) = P(\{1, 2\}) = \frac{2}{6} \]
    \[ P(\{1, 2, 3, 4\}) - P(\{3, 4, 5, 6\}) = P(\{1, 2\}) = \frac{4}{6} - \frac{4}{6} = 0 \]
    \[ \frac{2}{6} \neq 0 \]

\end{document}
