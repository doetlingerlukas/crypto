% !TEX TS-program = xelatex
%
\documentclass[11pt,a4paper]{article}

\usepackage{lmodern}
\usepackage{amsmath}
\usepackage{amssymb}
\usepackage[ngerman]{babel}
\usepackage[procnames]{listings}
\usepackage{listings-rust}
\usepackage{listings}
\usepackage{ulem}
\usepackage{amsthm}
\usepackage{tikz}
\usepackage{wasysym}
\usepackage[binary-units=true]{siunitx}
\DeclareSIPrefix\nona{N}{10^27}

\usepackage{multicol}
\usepackage{enumitem}
\setenumerate{nolistsep, itemsep=.5em}
\setitemize{nolistsep, itemsep=.5em}

\setlength{\parindent}{0in}
\setlength{\parskip}{.5em}

\author{}
\date{}

\definecolor{red}{HTML}{f92672}
\definecolor{green}{HTML}{009900}

\lstloadlanguages{Matlab}
\lstloadlanguages{Ruby}

\lstset{%
  basicstyle = \ttfamily\color{black},
  commentstyle = \ttfamily\color{gray},
  keywordstyle = \ttfamily\color{red},
  stringstyle = \color{orange},
  breaklines = true,
  showstringspaces = false,
  procnamekeys={def},
  procnamestyle=\color{green}
}


\begin{document}
  \title{Informationstheorie und Kryptologie: 9. Blatt für 24.5.2018}
  \maketitle

  \section*{25)}

  \section*{26)}

  \begin{multicols}{2}
    Wahrscheinlichkeiten:

    \begin{itemize}
      \item $P\{M = m_1\} = \frac{1}{2}$
      \item $P\{M = m_2\} = \frac{1}{4}$
      \item $P\{M = m_3\} = \frac{1}{4}$
    \end{itemize}

    \columnbreak

    Schlüssel:

    \begin{itemize}
      \item $P\{K = k_1\} = \frac{3}{4}$
      \item $P\{K = k_2\} = \frac{1}{4}$
    \end{itemize}
  \end{multicols}

  \begin{multicols}{2}
    Verschlüsselungstabelle:

    \begin{tabular}{|*3{c|}}
      \hline
      $e$ & $k_1$ & $k_2$ \\
      \hline
      $m_1$ & $c_1$ & $c_2$ \\
      $m_2$ & $c_2$ & $c_3$ \\
      $m_3$ & $c_3$ & $c_1$ \\
      \hline
    \end{tabular}

    \columnbreak

    Entschlüsselungstabelle:

    \begin{tabular}{|*3{c|}}
      \hline
      $d$ & $k_1$ & $k_2$ \\
      \hline
      $c_1$ & $m_1$ & $m_3$ \\
      $c_2$ & $m_2$ & $m_1$ \\
      $c_3$ & $m_3$ & $m_2$ \\
      \hline
    \end{tabular}
  \end{multicols}

  \begin{align*}
    P\{C = c_1\} = P\{M = m_1 | K = k_1\} + P\{M = m_3 | K = k_2\} = \frac{1}{2} \cdot \frac{3}{4} + \frac{1}{4} \cdot \frac{1}{4} = \frac{7}{16}\\
    P\{C = c_2\} = P\{M = m_2 | K = k_1\} + P\{M = m_1 | K = k_2\} = \frac{1}{4} \cdot \frac{3}{4} + \frac{1}{2} \cdot \frac{1}{4} = \frac{5}{16}\\
    P\{C = c_3\} = P\{M = m_3 | K = k_1\} + P\{M = m_2 | K = k_2\} = \frac{1}{4} \cdot \frac{3}{4} + \frac{1}{4} \cdot \frac{1}{4} = \frac{4}{16}\\
  \end{align*}

  \section*{27)}
\end{document}
