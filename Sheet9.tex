% !TEX TS-program = xelatex
%
\usepackage{lmodern}
\usepackage{amsmath}
\usepackage{amssymb}
\usepackage[ngerman]{babel}
\usepackage[procnames]{listings}
\usepackage{listings-rust}
\usepackage{listings}
\usepackage{ulem}
\usepackage{amsthm}
\usepackage{tikz}
\usepackage{wasysym}

\author{}

\lstloadlanguages{Matlab}
\lstloadlanguages{Ruby}

\lstset{%
  basicstyle = \ttfamily\color{black},
  commentstyle = \ttfamily\color{gray},
  keywordstyle = \ttfamily\color{red},
  stringstyle = \color{orange},
  breaklines = true,
  showstringspaces = false,
  procnamekeys={def},
  procnamestyle=\color{green}
}


\begin{document}
  \title{Informationstheorie und Kryptologie: 9. Blatt für 24.5.2018}
  \maketitle

  \section*{25)}

  \section*{26)}

  \begin{multicols}{2}
    Wahrscheinlichkeiten:

    \begin{itemize}
      \item $P\{M = m_1\} = \frac{1}{2}$
      \item $P\{M = m_2\} = \frac{1}{4}$
      \item $P\{M = m_3\} = \frac{1}{4}$
    \end{itemize}

    \columnbreak

    Schlüssel:

    \begin{itemize}
      \item $P\{K = k_1\} = \frac{3}{4}$
      \item $P\{K = k_2\} = \frac{1}{4}$
    \end{itemize}
  \end{multicols}

  \begin{multicols}{2}
    Verschlüsselungstabelle:

    \begin{tabular}{|*3{c|}}
      \hline
      $e$ & $k_1$ & $k_2$ \\
      \hline
      $m_1$ & $c_1$ & $c_2$ \\
      $m_2$ & $c_2$ & $c_3$ \\
      $m_3$ & $c_3$ & $c_1$ \\
      \hline
    \end{tabular}

    \columnbreak

    Entschlüsselungstabelle:

    \begin{tabular}{|*3{c|}}
      \hline
      $d$ & $k_1$ & $k_2$ \\
      \hline
      $c_1$ & $m_1$ & $m_3$ \\
      $c_2$ & $m_2$ & $m_1$ \\
      $c_3$ & $m_3$ & $m_2$ \\
      \hline
    \end{tabular}
  \end{multicols}

  \begin{align*}
    & P\{C = c_1\} = P\{M = m_1, K = k_1\} + P\{M = m_3, K = k_2\} = \frac{1}{2} \cdot \frac{3}{4} + \frac{1}{4} \cdot \frac{1}{4} = \frac{7}{16}\\
    & P\{C = c_2\} = P\{M = m_2, K = k_1\} + P\{M = m_1, K = k_2\} = \frac{1}{4} \cdot \frac{3}{4} + \frac{1}{2} \cdot \frac{1}{4} = \frac{5}{16}\\
    & P\{C = c_3\} = P\{M = m_3, K = k_1\} + P\{M = m_2, K = k_2\} = \frac{1}{4} \cdot \frac{3}{4} + \frac{1}{4} \cdot \frac{1}{4} = \frac{1}{4}\\
  \end{align*}

  \begin{align*}
    & P\{M = m_1 | C = c_1\} = P\{M = m_1, K = k_1\} \div P\{C = c_1\} = \frac{1}{2} \cdot \frac{3}{4} \div \frac{7}{16} = \frac{6}{7}\\
    & P\{M = m_2 | C = c_1\} = 0 \div P\{C = c_1\} = 0\\
    & P\{M = m_3 | C = c_1\} = P\{M = m_3, K = k_2\} \div P\{C = c_1\} = \frac{1}{4} \cdot \frac{1}{4} \div \frac{7}{16} = \frac{1}{7}\\
    & P\{M = m_1 | C = c_2\} = P\{M = m_1, K = k_2\} \div P\{C = c_2\} = \frac{1}{2} \cdot \frac{1}{4} \div \frac{5}{16} = \frac{2}{5}\\
    & P\{M = m_2 | C = c_2\} = P\{M = m_2, K = k_1\} \div P\{C = c_2\} = \frac{1}{4} \cdot \frac{3}{4} \div \frac{5}{16} = \frac{3}{5}\\
    & P\{M = m_3 | C = c_2\} = 0 \div P\{C = c_2\} = 0\\
    & P\{M = m_1 | C = c_3\} = 0 \div P\{C = c_3\} = 0\\
    & P\{M = m_2 | C = c_3\} = P\{M = m_2, K = k_2\} \div P\{C = c_3\} = \frac{1}{4} \cdot \frac{1}{4} \div \frac{1}{4} = \frac{1}{4}\\
    & P\{M = m_3 | C = c_3\} = P\{M = m_3, K = k_1\} \div P\{C = c_3\} = \frac{1}{4} \cdot \frac{3}{4} \div \frac{1}{4} = \frac{3}{4}\\
  \end{align*}

  \begin{align*}
    & P\{K = k_1 | C = c_1\} = P\{M = m_1, K = k_1\} \div P\{C = c_1\} = \frac{1}{2} \cdot \frac{3}{4} \div \frac{7}{16} = \frac{6}{7}\\
    & P\{K = k_2 | C = c_1\} = P\{M = m_3, K = k_2\} \div P\{C = c_1\} = \frac{1}{4} \cdot \frac{1}{4} \div \frac{7}{16} = \frac{1}{7}\\
    & P\{K = k_1 | C = c_2\} = P\{M = m_2, K = k_1\} \div P\{C = c_2\} = \frac{1}{4} \cdot \frac{3}{4} \div \frac{5}{16} = \frac{3}{5}\\
    & P\{K = k_2 | C = c_2\} = P\{M = m_1, K = k_2\} \div P\{C = c_2\} = \frac{1}{2} \cdot \frac{1}{4} \div \frac{5}{16} = \frac{2}{5}\\
    & P\{K = k_1 | C = c_3\} = P\{M = m_3, K = k_1\} \div P\{C = c_3\} = \frac{1}{4} \cdot \frac{3}{4} \div \frac{1}{4} = \frac{3}{4}\\
    & P\{K = k_2 | C = c_3\} = P\{M = m_2, K = k_2\} \div P\{C = c_3\} = \frac{1}{4} \cdot \frac{1}{4} \div \frac{1}{4} = \frac{1}{4}\\
  \end{align*}

  \section*{27)}
\end{document}
