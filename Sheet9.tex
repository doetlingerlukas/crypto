% !TEX TS-program = xelatex
%
\usepackage{lmodern}
\usepackage{amsmath}
\usepackage{amssymb}
\usepackage[ngerman]{babel}
\usepackage[procnames]{listings}
\usepackage{listings-rust}
\usepackage{listings}
\usepackage{ulem}
\usepackage{amsthm}
\usepackage{tikz}
\usepackage{wasysym}

\author{}

\lstloadlanguages{Matlab}
\lstloadlanguages{Ruby}

\lstset{%
  basicstyle = \ttfamily\color{black},
  commentstyle = \ttfamily\color{gray},
  keywordstyle = \ttfamily\color{red},
  stringstyle = \color{orange},
  breaklines = true,
  showstringspaces = false,
  procnamekeys={def},
  procnamestyle=\color{green}
}


\begin{document}
  \title{Informationstheorie und Kryptologie: 9. Blatt für 24.5.2018}
  \maketitle

  \section*{25)}

    \subsection*{a)}

      \underline{0} \underline{1} \underline{011} \underline{00} \underline{1000} \underline{111} \underline{1} \\

      $(0,0,0), (0,0,1), (2,2,1), (3,1,0), (3,3,0), (8,2,1), (1,0,1)$\\

      Kompressionsrate: $\frac{7}{15} = 0,46\dot{6}$

    \subsection*{b)}

      \underline{0} \underline{00000000\ldots}\\

      $(0,0,0), (1,n-2,0)$\\

      Kompressionsrate: $\frac{2}{n} = \lim\limits_{n \rightarrow \infty}{0}$

    \subsection*{c)}

      \underline{0} \underline{1} \underline{00} \underline{10010010010010\ldots}\\

      $(0,0,0), (0,0,1), (2,1,0), (3,3 \cdot (n - 1) - 1,0)$\\

      Kompressionsrate: $\frac{4}{3n} = \lim\limits_{n \rightarrow \infty}{0}$

  \section*{26)}

  \begin{multicols}{2}
    Wahrscheinlichkeiten:

    \begin{itemize}
      \item $P\{M = m_1\} = \frac{1}{2}$
      \item $P\{M = m_2\} = \frac{1}{4}$
      \item $P\{M = m_3\} = \frac{1}{4}$
    \end{itemize}

    \columnbreak

    Schlüssel:

    \begin{itemize}
      \item $P\{K = k_1\} = \frac{3}{4}$
      \item $P\{K = k_2\} = \frac{1}{4}$
    \end{itemize}
  \end{multicols}

  \begin{multicols}{2}
    Verschlüsselungstabelle:

    \begin{tabular}{|*3{c|}}
      \hline
      $e$ & $k_1$ & $k_2$ \\
      \hline
      $m_1$ & $c_1$ & $c_2$ \\
      $m_2$ & $c_2$ & $c_3$ \\
      $m_3$ & $c_3$ & $c_1$ \\
      \hline
    \end{tabular}

    \columnbreak

    Entschlüsselungstabelle:

    \begin{tabular}{|*3{c|}}
      \hline
      $d$ & $k_1$ & $k_2$ \\
      \hline
      $c_1$ & $m_1$ & $m_3$ \\
      $c_2$ & $m_2$ & $m_1$ \\
      $c_3$ & $m_3$ & $m_2$ \\
      \hline
    \end{tabular}
  \end{multicols}

  \begin{align*}
    & P\{C = c_1\} = P\{M = m_1, K = k_1\} + P\{M = m_3, K = k_2\} = \frac{1}{2} \cdot \frac{3}{4} + \frac{1}{4} \cdot \frac{1}{4} = \frac{7}{16}\\
    & P\{C = c_2\} = P\{M = m_2, K = k_1\} + P\{M = m_1, K = k_2\} = \frac{1}{4} \cdot \frac{3}{4} + \frac{1}{2} \cdot \frac{1}{4} = \frac{5}{16}\\
    & P\{C = c_3\} = P\{M = m_3, K = k_1\} + P\{M = m_2, K = k_2\} = \frac{1}{4} \cdot \frac{3}{4} + \frac{1}{4} \cdot \frac{1}{4} = \frac{1}{4}\\
  \end{align*}

  \begin{align*}
    & P\{M = m_1 | C = c_1\} = P\{M = m_1, K = k_1\} \div P\{C = c_1\} = \frac{1}{2} \cdot \frac{3}{4} \div \frac{7}{16} = \frac{6}{7}\\
    & P\{M = m_2 | C = c_1\} = 0 \div P\{C = c_1\} = 0\\
    & P\{M = m_3 | C = c_1\} = P\{M = m_3, K = k_2\} \div P\{C = c_1\} = \frac{1}{4} \cdot \frac{1}{4} \div \frac{7}{16} = \frac{1}{7}\\
    & P\{M = m_1 | C = c_2\} = P\{M = m_1, K = k_2\} \div P\{C = c_2\} = \frac{1}{2} \cdot \frac{1}{4} \div \frac{5}{16} = \frac{2}{5}\\
    & P\{M = m_2 | C = c_2\} = P\{M = m_2, K = k_1\} \div P\{C = c_2\} = \frac{1}{4} \cdot \frac{3}{4} \div \frac{5}{16} = \frac{3}{5}\\
    & P\{M = m_3 | C = c_2\} = 0 \div P\{C = c_2\} = 0\\
    & P\{M = m_1 | C = c_3\} = 0 \div P\{C = c_3\} = 0\\
    & P\{M = m_2 | C = c_3\} = P\{M = m_2, K = k_2\} \div P\{C = c_3\} = \frac{1}{4} \cdot \frac{1}{4} \div \frac{1}{4} = \frac{1}{4}\\
    & P\{M = m_3 | C = c_3\} = P\{M = m_3, K = k_1\} \div P\{C = c_3\} = \frac{1}{4} \cdot \frac{3}{4} \div \frac{1}{4} = \frac{3}{4}\\
  \end{align*}

  \begin{align*}
    & P\{K = k_1 | C = c_1\} = P\{M = m_1, K = k_1\} \div P\{C = c_1\} = \frac{1}{2} \cdot \frac{3}{4} \div \frac{7}{16} = \frac{6}{7}\\
    & P\{K = k_2 | C = c_1\} = P\{M = m_3, K = k_2\} \div P\{C = c_1\} = \frac{1}{4} \cdot \frac{1}{4} \div \frac{7}{16} = \frac{1}{7}\\
    & P\{K = k_1 | C = c_2\} = P\{M = m_2, K = k_1\} \div P\{C = c_2\} = \frac{1}{4} \cdot \frac{3}{4} \div \frac{5}{16} = \frac{3}{5}\\
    & P\{K = k_2 | C = c_2\} = P\{M = m_1, K = k_2\} \div P\{C = c_2\} = \frac{1}{2} \cdot \frac{1}{4} \div \frac{5}{16} = \frac{2}{5}\\
    & P\{K = k_1 | C = c_3\} = P\{M = m_3, K = k_1\} \div P\{C = c_3\} = \frac{1}{4} \cdot \frac{3}{4} \div \frac{1}{4} = \frac{3}{4}\\
    & P\{K = k_2 | C = c_3\} = P\{M = m_2, K = k_2\} \div P\{C = c_3\} = \frac{1}{4} \cdot \frac{1}{4} \div \frac{1}{4} = \frac{1}{4}\\
  \end{align*}

  \section*{27)}

  Nachrichtenmehrdeutigkeit:

  \begin{align*}
    & H(M | C) & = &\ P\{C = c_1\} \cdot H(M | C = c_1)\\
    &          & + &\ P\{C = c_2\} \cdot H(M | C = c_2)\\
    &          & + &\ P\{C = c_3\} \cdot H(M | C = c_3)\\
    &          & = &\ \frac{7}{16} \cdot H(\frac{6}{7}, 0,  \frac{1}{7}) + \frac{5}{16} \cdot H(\frac{2}{5}, \frac{3}{5}, 0) + \frac{1}{4} \cdot H(0, \frac{1}{4}, \frac{3}{4})\\
    &          & \approx &\ \SI{0,77}{\bit}
  \end{align*}

  Schlüsselmehrdeutigkeit:

  \begin{align*}
    & H(K | C) & = &\ P\{C = c_1\} \cdot H(K | C = c_1)\\
    &          & + &\ P\{C = c_2\} \cdot H(K | C = c_2)\\
    &          & + &\ P\{C = c_3\} \cdot H(K | C = c_3)\\
    &          & = &\ \frac{7}{16} \cdot H(\frac{6}{7}, \frac{1}{7}) + \frac{5}{16} \cdot H(\frac{3}{5}, \frac{2}{5}) + \frac{1}{4} \cdot H(\frac{3}{4}, \frac{1}{4})\\
    &          & \approx &\ \SI{0,77}{\bit}
  \end{align*}

  Welche Informationen erhalten wir aus dem Chiffretext?

  \begin{quote}
    Satz 2.5:

    \begin{align*}
      & I(X | Y) = H(X) - H(X | Y) = H(X) + H(Y) - H(X, Y)
    \end{align*}
  \end{quote}

  \begin{align*}
    & I(M | C) = H(M) - H(M | C) = H(\frac{1}{2}, \frac{1}{4}, \frac{1}{4}) - H(M | C) \approx \SI{0,735}{\bit}
  \end{align*}

  \begin{align*}
    & I(K | C) = H(K) - H(K | C) = H(\frac{3}{4}, \frac{1}{4}) - H(K | C) \approx \SI{0,046}{\bit}
  \end{align*}

  Wie würden Sie einen Maximum-Likelihood-Angriff führen?

  Chiffretext $c$ wird abgehört, dann wird die Nachricht, für die $P\{M = m | C = c\}$ maximal ist, ausgewählt. In diesem Fall also $c_1 \rightarrow m_1, c_2 \rightarrow m_2, c_3 \rightarrow m_3$.
\end{document}
