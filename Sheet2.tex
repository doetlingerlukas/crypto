\documentclass[11pt,a4paper]{article}


\usepackage{amsmath}
\usepackage[ngerman]{babel}
\usepackage[utf8x]{inputenx}
\usepackage[T1]{fontenc}
\usepackage{lmodern}
\usepackage{listings}
\lstset{language=MatLab}


\begin{document}
	\title{InformationsTheorie und Kryptologie: 2.Blatt für 15.3.2018}
	\maketitle
	
	\section*{4)}
	\section*{5)}
	Wie groß ist die Wahrscheinlichkeit, beim Lotto "6 aus 45":
	\begin{itemize}
		\item einen Dreier
		\item einen Vierer
		\item einen Fünfer
		\item einen Sechser
	\end{itemize}
	zu tippen? Überprüfen Sie Ihre Berechnung mit einer Geeigneten odf von GNu Octave oder MatLab.\\
	
	\textbf{Binomialkoeffizient} für die verschiedenen Möglichkeiten 6 Zahlen aus 45 zu ziehen.\\\\
	\(M=\binom{45}{6} =\frac{45!}{(45-6)! * 6!}= 8.145.060\)\\\\
	a) P(Dreier)= \( \frac{\binom{6}{3} * \binom{39}{3}}{\binom{45}{6}}=\frac{20*9139}{8.145.060}=0.022406\)\\
	b) P(Vierer)= \( \frac{\binom{6}{4} * \binom{39}{2}}{\binom{45}{6}}=\frac{15*741}{8.145.060}=0.001365\)\\
	c) P(Fünfer)= \( \frac{\binom{6}{5} * \binom{39}{1}}{\binom{45}{6}}=\frac{6*39}{8.145.060}=0.00002873\)\\
	d) P(Sechser)= \( \frac{\binom{6}{6} * 1}{\binom{45}{6}}=\frac{1}{8.145.060}=0.000000122773\)\\\\
	PDF für berechnung:\\
	\begin{lstlisting}[frame=single]
		p3= hygepdf(3,45,6,6)
		p4= hygepdf(4,45,6,6)
		p5= hygepdf(5,45,6,6)
		p6= hygepdf(6,45,6,6)
	\end{lstlisting}
	Output:\\
	p3 =  0.022441\\
	p4 =  0.0013646\\
	p5 =   2.8729e-005\\
	p6 =   1.2277e-007\\

	\section*{6)}
\end{document}