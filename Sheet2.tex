% !TEX TS-program = xelatex
%
\usepackage{lmodern}
\usepackage{amsmath}
\usepackage{amssymb}
\usepackage[ngerman]{babel}
\usepackage[procnames]{listings}
\usepackage{listings-rust}
\usepackage{listings}
\usepackage{ulem}
\usepackage{amsthm}
\usepackage{tikz}
\usepackage{wasysym}

\author{}

\lstloadlanguages{Matlab}
\lstloadlanguages{Ruby}

\lstset{%
  basicstyle = \ttfamily\color{black},
  commentstyle = \ttfamily\color{gray},
  keywordstyle = \ttfamily\color{red},
  stringstyle = \color{orange},
  breaklines = true,
  showstringspaces = false,
  procnamekeys={def},
  procnamestyle=\color{green}
}


\newcommand*\circled[1]{\tikz[baseline=(char.base)]{
            \node[shape=circle,draw,inner sep=2pt] (char) {#1};}}

\begin{document}
	\title{Informationstheorie und Kryptologie: 2. Blatt für 15.3.2018}
	\maketitle

	\section*{4)}
		\begin{center}
			\def\arraystretch{1.5}%
			\begin{tabular}{ l l l l l | c c c c c | c | c }
				$r_0$          & $r_1$           & $r_2$           & $r_3$           & $r_4$           & $a_0$ & $a_1$ & $a_2$ & $a_3$ & $a_4$ & H              & L \\
				\hline
				$\frac{5}{16}$ & $\frac{15}{16}$ & $\frac{25}{16}$ & $\frac{20}{16}$ & $\frac{15}{16}$ & $x_0$ & $x_1$ & $x_2$ & $x_3$ & $x_4$ & \circled{2}, 3 & \circled{0}, 1, 4 \\
				               &                 & $\frac{14}{16}$ &                 &                 & $x_2$ &       &       &       &       & \circled{3}    & \circled{1}, 2, 4 \\
				               &                 &                 & $\frac{19}{16}$ &                 &       & $x_3$ &       &       &       & \circled{3}    & \circled{2}, 4 \\
				               &                 &                 & $\frac{17}{16}$ &                 &       &       & $x_3$ &       &       & \circled{3}    & \circled{4} \\
				               &                 &                 & $\frac{16}{16}$ &                 &       &       &       &       & $x_3$ &                & 3
			\end{tabular}
		\end{center}

		\[ r_k = r_k - (1 - r_j) \textit{ and } a_j = x_k \]
		\begin{center}
			\def\arraystretch{1.5}%
			\begin{tabular}{c c | l }
				1) $k = 2 , j = 0$ & $r_2 = \frac{25}{16} - (1 - \frac{5}{16}) = \frac{25}{16} - \frac{11}{16} = \frac{14}{16}$ & $\frac{14}{16} \leq 1 \Rightarrow H = 3, L = 1, 2, 4$ \\
				2) $k = 3 , j = 1$ & $r_3 = \frac{20}{16} - (1 - \frac{15}{16}) = \frac{20}{16} - \frac{1}{16} = \frac{19}{16}$ & $\frac{19}{16} > 1 \Rightarrow H = 3, L = 2, 4$ \\
				3) $k = 3 , j = 2$ & $r_3 = \frac{19}{16} - (1 - \frac{14}{16}) = \frac{19}{16} - \frac{2}{16} = \frac{17}{16}$ & $\frac{17}{16} > 1 \Rightarrow H = 3, L = 4$ \\
				4) $k = 3 , j = 4$ & $r_3 = \frac{17}{16} - (1 - \frac{15}{16}) = \frac{17}{16} - \frac{1}{16} = \frac{16}{16}$ & $\frac{16}{16} \leq 1 \Rightarrow H = \emptyset, L = 3$ \\
			\end{tabular}
		\end{center}

		\begin{tabular}{cc}
      \begin{minipage}{.5\linewidth}
        \centering
				\begin{tikzpicture}
					\draw (0cm,0cm) -- (5.5cm,0cm);
					\foreach \x in {0,1,2,3,4,5}
					\draw (\x,-.1) -- (\x,.1) node[below=4pt] {$\scriptstyle\x$};

					\draw (-0.1cm,0cm) -- (-0.1cm,5cm);
					\draw (-0.1cm,0cm) -- (-0.2cm,0cm);
					\draw (-.1, 0.9375) -- (.1, 0.9375) node[left=4pt] {$\frac{5}{16}$};
					\draw (-.1, 1.875) -- (.1, 1.875) node[left=4pt] {$\frac{10}{16}$};
					\draw (-.1, 2.8125) -- (.1, 2.8125) node[left=4pt] {$\frac{15}{16}$};
					\draw (-.1, 2.8125) -- (.1, 2.8125) node[left=4pt] {$\frac{15}{16}$};
					\draw (-.1, 3.75) -- (.1, 3.75) node[left=4pt] {$\frac{20}{16}$};
					\draw (-.1, 4.6875) -- (.1, 4.6875) node[left=4pt] {$\frac{25}{16}$};

					\draw[gray!50, text=black] (-0.2 cm, 3 cm) -- (5.5 cm, 3 cm)
						node at (5.6 cm, 3 cm) {1};

					\foreach \x/\y in {0/0.9375, 1/2.8125, 2/4.6875, 3/3.75, 4/2.8125} {
						\draw (\x cm, 0cm) rectangle (1cm + \x cm,\y cm)
							node at (0.5cm + \x cm, 0.5cm) {$x_\x$};
					};
				\end{tikzpicture}
      \end{minipage}
      \begin{minipage}{.5\linewidth}
        \centering
				\begin{tikzpicture}

					\draw (0cm,0cm) -- (5.5cm,0cm);
					\foreach \x in {0,1,2,3,4,5}
					\draw (\x,-.1) -- (\x,.1) node[below=4pt] {$\scriptstyle\x$};

					\draw (-0.1cm,0cm) -- (-0.1cm,5cm);
					\draw (-0.1cm,0cm) -- (-0.2cm,0cm);
					\draw (-.1, 0.9375) -- (.1, 0.9375) node[left=4pt] {$\frac{5}{16}$};
					\draw (-.1, 1.875) -- (.1, 1.875) node[left=4pt] {$\frac{10}{16}$};
					\draw (-.1, 2.8125) -- (.1, 2.8125) node[left=4pt] {$\frac{15}{16}$};
					\draw (-.1, 2.8125) -- (.1, 2.8125) node[left=4pt] {$\frac{15}{16}$};
					\draw (-.1, 3.75) -- (.1, 3.75) node[left=4pt] {$\frac{20}{16}$};
					\draw (-.1, 4.6875) -- (.1, 4.6875) node[left=4pt] {$\frac{25}{16}$};

					\draw[gray!50, text=black] (-0.2 cm, 3 cm) -- (5.5 cm, 3 cm)
						node at (5.6 cm, 3 cm) {1};

					\foreach \x/\y in {0/0.9375, 1/2.8125, 2/2.625, 3/3, 4/2.8125} {
						\draw (\x cm, 0cm) rectangle (1cm + \x cm,\y cm)
							node at (0.5cm + \x cm, 0.5cm) {$x_\x$};
					};

					\draw (0 cm, 0cm) rectangle (1cm + 0 cm, 3 cm)
						node at (0.5cm, 2cm) {$x_2$};
					\draw (1 cm, 0cm) rectangle (1cm + 1 cm, 3 cm)
						node at (1.5cm, 2.85cm) {$x_3$};
					\draw (2 cm, 0cm) rectangle (1cm + 2 cm, 3 cm)
						node at (2.5cm, 2.8cm) {$x_3$};
					\draw (4 cm, 0cm) rectangle (1cm + 4 cm, 3 cm)
						node at (4.5cm, 2.85cm) {$x_3$};
				\end{tikzpicture}
      \end{minipage}
    \end{tabular}

	\section*{5)}
	Wie groß ist die Wahrscheinlichkeit, beim Lotto "6 aus 45":
	\begin{itemize}
		\item einen Dreier
		\item einen Vierer
		\item einen Fünfer
		\item einen Sechser
	\end{itemize}
	zu tippen? Überprüfen Sie Ihre Berechnung mit einer Geeigneten odf von GNU Octave oder MatLab.\\

	\textbf{Binomialkoeffizient} für die verschiedenen Möglichkeiten 6 Zahlen aus 45 zu ziehen.\\\\
	\(M=\binom{45}{6} =\frac{45!}{(45-6)! * 6!}= 8.145.060\)\\\\
	a) P(Dreier)= \( \frac{\binom{6}{3} * \binom{39}{3}}{\binom{45}{6}}=\frac{20*9139}{8.145.060}=0.022406\)\\
	b) P(Vierer)= \( \frac{\binom{6}{4} * \binom{39}{2}}{\binom{45}{6}}=\frac{15*741}{8.145.060}=0.001365\)\\
	c) P(Fünfer)= \( \frac{\binom{6}{5} * \binom{39}{1}}{\binom{45}{6}}=\frac{6*39}{8.145.060}=0.00002873\)\\
	d) P(Sechser)= \( \frac{\binom{6}{6} * 1}{\binom{45}{6}}=\frac{1}{8.145.060}=0.000000122773\)\\\\
	PDF für berechnung:\\
	\begin{lstlisting}[language=MatLab,frame=single]
		p3= hygepdf(3,45,6,6)
		p4= hygepdf(4,45,6,6)
		p5= hygepdf(5,45,6,6)
		p6= hygepdf(6,45,6,6)
	\end{lstlisting}
	Output:\\
	p3 =  0.022441\\
	p4 =  0.0013646\\
	p5 =   2.8729e-005\\
	p6 =   1.2277e-007\\

	\newpage
	\section*{6)}

  \begin{align*}
    \text{Ergebnisraum}\ \Omega = (0,1]
  \end{align*}

  a) Die Wahrscheinlichkeit, dass die Fläche kleiner als $\frac{1}{2}$ ist die gleiche wie die, dass die Seitenlänge kleiner als $\sqrt{\frac{1}{2}}$ ist:

  \begin{align*}
    P((0,\sqrt{\frac{1}{2}})) = \frac{\sqrt{\frac{1}{2}} - 0}{1} = \sqrt{\frac{1}{2}} \approx 70,7 \%
  \end{align*}

  b) Die Wahrscheinlichkeit, dass die Fläche größer als $\frac{1}{2}$ ist, ist die gleiche wie die, dass die Seitenlänge größer als $\sqrt{\frac{1}{2}}$ ist:

  \begin{align*}
    P((\sqrt{\frac{1}{2}},1]) = \frac{1 - \sqrt{\frac{1}{2}}}{1} = 1 - \sqrt{\frac{1}{2}} \approx 29,3 \%
  \end{align*}

  Da beide Intervalle für die Fläche gleich lang sind, aber nicht die gleiche Wahrscheinlichkeit haben, ist die Fläche nicht gleichverteilt.

\end{document}
