% !TEX TS-program = xelatex
%
\usepackage{lmodern}
\usepackage{amsmath}
\usepackage{amssymb}
\usepackage[ngerman]{babel}
\usepackage[procnames]{listings}
\usepackage{listings-rust}
\usepackage{listings}
\usepackage{ulem}
\usepackage{amsthm}
\usepackage{tikz}
\usepackage{wasysym}

\author{}

\lstloadlanguages{Matlab}
\lstloadlanguages{Ruby}

\lstset{%
  basicstyle = \ttfamily\color{black},
  commentstyle = \ttfamily\color{gray},
  keywordstyle = \ttfamily\color{red},
  stringstyle = \color{orange},
  breaklines = true,
  showstringspaces = false,
  procnamekeys={def},
  procnamestyle=\color{green}
}


\begin{document}
  \title{Informationstheorie und Kryptologie: 7. Blatt für 17.5.2018}
  \maketitle

  \section*{22)}

  \lstinputlisting[language=Ruby]{decode_natural_number.rb}

  \section*{23)}

  Entropie $H(\frac{1}{2}, \frac{1}{4}, \frac{1}{8}, \frac{1}{8}) = 1.75$\\
  erwartete Länge der Huffman-Codierung $L = 1.75$\\
  \\
  $p_0 = \frac{1}{2}, p_1 = \frac{1}{4}, p_3 = \frac{1}{8}, p_4 = \frac{1}{8}$\\
  \\
  \begin{align*}
    & L(\frac{1}{2}, \frac{1}{4}, \frac{1}{8}, \frac{1}{8}) = &\\
    & L(\frac{1}{2}, \frac{1}{4}, \frac{1}{4}) + \frac{1}{8} + \frac{1}{8} = L(\frac{1}{2}, \frac{1}{4}, \frac{1}{4}) + \frac{1}{4} =\\ 
    & L(\frac{1}{2}, \frac{1}{2}) + \frac{1}{4} + \frac{1}{4} + \frac{1}{4} = L(\frac{1}{2}, \frac{1}{2}) + \frac{3}{4} =\\
    & L(1) + \frac{1}{2} + \frac{1}{2} + \frac{3}{4} = L(1) + \frac{7}{4} = 0 + \frac{7}{4} = 1.75
  \end{align*}
  \\
  Der Vorteil der Rekursionsformel liegt darin, dass ma daraus die Huffman-Codierung erzeugen kann.\\
  \\
  Codierung: $p_0 \rightarrow 0, p_1 \rightarrow 10, p_2 \rightarrow 110, p_3 \rightarrow 111$\\
  \\
  \begin{align*}
    & 1 \cdot \frac{1}{2} + 2 \cdot \frac{1}{4} + 3 \cdot \frac{1}{8} + 3 \cdot \frac{1}{8} = 1.75
  \end{align*}

  \section*{24)}
\end{document}
