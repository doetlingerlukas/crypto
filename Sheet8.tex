% !TEX TS-program = xelatex
%
\usepackage{lmodern}
\usepackage{amsmath}
\usepackage{amssymb}
\usepackage[ngerman]{babel}
\usepackage[procnames]{listings}
\usepackage{listings-rust}
\usepackage{listings}
\usepackage{ulem}
\usepackage{amsthm}
\usepackage{tikz}
\usepackage{wasysym}

\author{}

\lstloadlanguages{Matlab}
\lstloadlanguages{Ruby}

\lstset{%
  basicstyle = \ttfamily\color{black},
  commentstyle = \ttfamily\color{gray},
  keywordstyle = \ttfamily\color{red},
  stringstyle = \color{orange},
  breaklines = true,
  showstringspaces = false,
  procnamekeys={def},
  procnamestyle=\color{green}
}


\begin{document}
  \title{Informationstheorie und Kryptologie: 7. Blatt für 17.5.2018}
  \maketitle

  \section*{22)}

  \lstinputlisting[language=Ruby]{decode_natural_number.rb}

  \section*{23)}

  Entropie $H(\frac{1}{2}, \frac{1}{4}, \frac{1}{8}, \frac{1}{8}) = 1.75$\\
  erwartete Länge der Huffman-Codierung $L = 1.75$\\
  \\
  $p_0 = \frac{1}{2}, p_1 = \frac{1}{4}, p_3 = \frac{1}{8}, p_4 = \frac{1}{8}$\\
  \\
  \begin{align*}
    & L(\frac{1}{2}, \frac{1}{4}, \frac{1}{8}, \frac{1}{8}) = &\\
    & L(\frac{1}{2}, \frac{1}{4}, \frac{1}{4}) + \frac{1}{8} + \frac{1}{8} = L(\frac{1}{2}, \frac{1}{4}, \frac{1}{4}) + \frac{1}{4} =\\ 
    & L(\frac{1}{2}, \frac{1}{2}) + \frac{1}{4} + \frac{1}{4} + \frac{1}{4} = L(\frac{1}{2}, \frac{1}{2}) + \frac{3}{4} =\\
    & L(1) + \frac{1}{2} + \frac{1}{2} + \frac{3}{4} = L(1) + \frac{7}{4} = 0 + \frac{7}{4} = 1.75
  \end{align*}
  \\
  Der Vorteil der Rekursionsformel liegt darin, dass ma daraus die Huffman-Codierung erzeugen kann.\\
  \\
  Codierung: $p_0 \rightarrow 0, p_1 \rightarrow 10, p_2 \rightarrow 110, p_3 \rightarrow 111$\\
  \\
  \begin{align*}
    & 1 \cdot \frac{1}{2} + 2 \cdot \frac{1}{4} + 3 \cdot \frac{1}{8} + 3 \cdot \frac{1}{8} = 1.75
  \end{align*}

  \newpage

  \section*{24)}

  Wahrscheinlichkeiten: $\text{a} := \frac{1}{5}, \text{b} := \frac{1}{5}, \text{c} := \frac{3}{5}$

  \subsection*{a)}

  {
    \renewcommand{\arraystretch}{1.25}
    \begin{tabular}{l|l|l}
      $\text{c} = \frac{3}{5}$ &  $\text{c} = \frac{3}{5}$      &  $\text{(c, (a, b))} = 1$\\
      $\text{a} = \frac{1}{5}$ &  $\text{(a, b)} = \frac{2}{5}$ &  \\
      $\text{b} = \frac{1}{5}$ &
    \end{tabular}\\
    \\
    \\
    \begin{tabular}{l}
      $\text{a} \rightarrow 10$\\
      $\text{b} \rightarrow 11$\\
      $\text{c} \rightarrow 0$
    \end{tabular}\\
  }
  \\
  Kompressionsrate: $(2 \cdot \frac{1}{5} + 2 \cdot \frac{1}{5} + 1 \cdot \frac{3}{5}) / 1 = 1.4$

  \newpage

  \subsection*{b)}

  {
    \renewcommand{\arraystretch}{1.25}
    \begin{tabular}{l|l|l|l}
      $\text{cc} = \frac{9}{25}$ & $\text{cc} = \frac{9}{25}$       & $\text{cc} = \frac{9}{25}$       & $\text{cc} = \frac{9}{25}$                   \\
      $\text{ac} = \frac{3}{25}$ & $\text{ac} = \frac{3}{25}$       & $\text{ac} = \frac{3}{25}$       & $\text{((ba, bb), (aa, ab))} = \frac{4}{25}$ \\
      $\text{bc} = \frac{3}{25}$ & $\text{bc} = \frac{3}{25}$       & $\text{bc} = \frac{3}{25}$       & $\text{ac} = \frac{3}{25}$                   \\
      $\text{ca} = \frac{3}{25}$ & $\text{ca} = \frac{3}{25}$       & $\text{ca} = \frac{3}{25}$       & $\text{bc} = \frac{3}{25}$                   \\
      $\text{cb} = \frac{3}{25}$ & $\text{cb} = \frac{3}{25}$       & $\text{cb} = \frac{3}{25}$       & $\text{ca} = \frac{3}{25}$                   \\
      $\text{aa} = \frac{1}{25}$ & $\text{(ba, bb)} = \frac{2}{25}$ & $\text{(ba, bb)} = \frac{2}{25}$ & $\text{cb} = \frac{3}{25}$                   \\
      $\text{ab} = \frac{1}{25}$ & $\text{aa} = \frac{1}{25}$       & $\text{(aa, ab)} = \frac{2}{25}$ &                                              \\
      $\text{ba} = \frac{1}{25}$ & $\text{ab} = \frac{1}{25}$       &                                  &                                              \\
      $\text{bb} = \frac{1}{25}$ &                                  &                                  &                                              \\
    \end{tabular}\\
    \\
    \\
    \begin{tabular}{l|l|l}
      $\text{cc} = \frac{9}{25}$                   & $\text{cc} = \frac{9}{25}$                   & $\text{((ac, bc), ((ba, bb), (aa, ab)))} = \frac{10}{25}$ \\
      $\text{(ca, cb)} = \frac{6}{25}$             & $\text{(ca, cb)} = \frac{6}{25}$             & $\text{cc} = \frac{9}{25}$                                \\
      $\text{((ba, bb), (aa, ab))} = \frac{4}{25}$ & $\text{(ac, bc)} = \frac{6}{25}$             & $\text{(ca, cb)} = \frac{6}{25}$                          \\
      $\text{ac} = \frac{3}{25}$                   & $\text{((ba, bb), (aa, ab))} = \frac{4}{25}$ &                                                           \\
      $\text{bc} = \frac{3}{25}$                   &                                              &                                                           \\
    \end{tabular}\\
    \\
    \\
    \begin{tabular}{l|l}
      $\text{(cc, (ca, cb))} = \frac{15}{25}$                   & $\text{((cc, (ca, cb)), ((ac, bc), ((ba, bb), (aa, ab))))} = \frac{25}{25}$   \\
      $\text{((ac, bc), ((ba, bb), (aa, ab)))} = \frac{10}{25}$ & \\
    \end{tabular}\\
    \\
    \\
    \begin{tabular}{l}
      $\text{aa} \rightarrow 1110$ \\
      $\text{ab} \rightarrow 1111$ \\
      $\text{ac} \rightarrow 100$ \\
      $\text{ba} \rightarrow 1100$ \\
      $\text{bb} \rightarrow 1101$ \\
      $\text{bc} \rightarrow 101$ \\
      $\text{ca} \rightarrow 010$ \\
      $\text{cb} \rightarrow 011$ \\
      $\text{cc} \rightarrow 00$ \\
    \end{tabular}\\
  }
  \\
  Kompressionsrate:\\
  \begin{align*}
    (4 \cdot \frac{1}{25} + 4 \cdot \frac{1}{25} + 3 \cdot \frac{3}{25} + 4 \cdot \frac{1}{25} + 4 \cdot \frac{1}{25} + 3 \cdot \frac{3}{25} + 3 \cdot \frac{3}{25} + 3 \cdot \frac{3}{25} + 2 \cdot \frac{9}{25}) / 2 =\\
    (4 \cdot \frac{4}{25} + 4 \cdot \frac{9}{25} + \frac{18}{25}) / 2 =\\
    1.4
  \end{align*}

  \subsection*{c)}

  Es gilt\\
  \\
  \begin{align*}
    H \leq E(l(h(X0, X1, \cdots, X_k−1))) / k \leq H + \frac{1}{k}
  \end{align*}\\
  \\
  Für a) gilt also\\
  \\
  \begin{align*}
    1.3709505944546687 \leq 1.4 \leq 1.3709505944546687 + \frac{1}{1} = 2.3709505944546687
  \end{align*}\\
  \\
  und für b) gilt\\
  \\
  \begin{align*}
    1.3709505944546687 \leq 1.4 \leq 1.3709505944546687 + \frac{1}{2} = 1.8709505945
  \end{align*}


\end{document}
