% !TEX TS-program = xelatex
%
\usepackage{lmodern}
\usepackage{amsmath}
\usepackage{amssymb}
\usepackage[ngerman]{babel}
\usepackage[procnames]{listings}
\usepackage{listings-rust}
\usepackage{listings}
\usepackage{ulem}
\usepackage{amsthm}
\usepackage{tikz}
\usepackage{wasysym}

\author{}

\lstloadlanguages{Matlab}
\lstloadlanguages{Ruby}

\lstset{%
  basicstyle = \ttfamily\color{black},
  commentstyle = \ttfamily\color{gray},
  keywordstyle = \ttfamily\color{red},
  stringstyle = \color{orange},
  breaklines = true,
  showstringspaces = false,
  procnamekeys={def},
  procnamestyle=\color{green}
}


\begin{document}
  \title{Informationstheorie und Kryptologie: 6. Blatt für 26.4.2018}
  \maketitle

  \section*{16)}

  \subsection*{a)}

  \texttt{f c t s s t r n g r t h n f c t n} \\
  \\
  \texttt{f\underline{\ }ct \underline{\ }s str\underline{\ }nger\ \ \ \ th\underline{\ }n\ \ f\underline{\ }ct\underline{\ }n} \\
  \\
  \texttt{f\underline{a}ct \underline{i}s str\underline{a}nger\ \ \ \ th\underline{a}n\ \ f\underline{i}ct\underline{io}n} \\
  \\
  The original sentence was “Fact is stranger than fiction.” \\
  \\
  Compression rate: $\frac{17}{29} \approx 59\% $

  \subsection*{b)}

  \texttt{m t e a i s i \textvisiblespace{ }\textvisiblespace{ }r t y e a t s i n e} \\
  \\
  \texttt{m\underline{\ }t\underline{\ }e\underline{\ }a\underline{\ }i\underline{\ }s\underline{\ }i\underline{\ }\textvisiblespace\underline{\ }\textvisiblespace\underline{\ }r\underline{\ }t\underline{\ }y\underline{\ }e\underline{\ }a\underline{\ }t\underline{\ }s\underline{\ }i\underline{\ }n\underline{\ }e} \\
  \\
  \texttt{m\underline{a}t\underline{h}e\underline{m}a\underline{t}i\underline{c}s\underline{\ }i\underline{s}\textvisiblespace\underline{a}\textvisiblespace\underline{p}r\underline{e}t\underline{t}y\underline{\ }e\underline{x}a\underline{c}t\underline{\ }s\underline{c}i\underline{e}n\underline{c}e} \\
  \\
  The original sentence was “Mathematics is a pretty exact science.” \\
  \\
  Compression rate: $\frac{19}{37} \approx 51\% $

  \subsection*{c)}

  For text files (written in English), the expected compression rate is between 15\% and 20\%.

  \section*{17)}
  \section*{18)}

    \subsection*{a)}

      Gesamtzahl der Zeichenketten der Länge 20:\\
      \[27^{20} = 4.24 \cdot 10^{28} \]

    \subsection*{b)}

      \textbf{Satz 2.10}: \textit{Für eine ergodische Quelle mit Alphabet A und Entropie H, gilt für die Anzahl der typischen Wörter T}:
      \[ \#(T) \approx 2^{nH} \]
      \newline
      Geschätzte Anzahl sinnvoller Zeichenketten der Länge 20:
      \[ 2^{20 \cdot 1.5} = 1.07 \cdot 10^{9} \]

    \subsection*{c)}

      \begin{equation}
      	\begin{split}
      	  a) \rightarrow 4.24 \cdot 10^{28} \cdot \SI{20}{\byte} & = \SI{8.48}{\giga\byte} \cdot 10^{20} \\
      	  & = \SI{8.48}{\peta\byte} \cdot 10^{14} = \SI{848}{\nona\byte} \\
      	  b) \rightarrow 1.07 \cdot 10^{9} \cdot \SI{20}{\byte} & = \SI{21.4}{\giga\byte} \\
      	\end{split}
      \end{equation}
      \newline
      Alle Zeichenketten aus a) kann man nicht auf modernen Festplatten abspeichern, die Zeichenketten aus b) jedoch schon.\\

    \subsection*{d)}

      \begin{equation}
      	\begin{split}
      	  \SI{21.4}{\giga\byte} \cdot 0.75 & = \SI{16.05}{\giga\byte} \\
      	  & \neq \\
      	  \SI{8.48}{\giga\byte} \cdot 10^{19} \cdot 0.25 & = \SI{2.12}{\giga\byte} \cdot 10^{19} = \SI{21.2}{\nona\byte}
      	\end{split}
      \end{equation}
      \newline
      Diese Schlussfolgerung ist falsch.

\end{document}
