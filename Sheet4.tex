\documentclass[11pt,a4paper]{article}

\usepackage{amsmath}
\usepackage[ngerman]{babel}
\usepackage[utf8x]{inputenx}
\usepackage[T1]{fontenc}
\usepackage{lmodern}
\usepackage{listings}
\usepackage{ulem}
\usepackage{amsthm}
\usepackage{tikz}

\lstset{language=MatLab}
\author{}

\begin{document}
  \title{InformationsTheorie und Kryptologie: 4.Blatt für 12.4.2018}
  \maketitle
  
  \section*{11)}
  
    Quick-Sort in der \textit{Median-of-Three} Variante wählt zufällig 3 Werte aus der zu sortierenden Menge aus. Der Median dieser Werte wird als Pivot gewählt.\\
    
    \subsection*{a)}
      
      Die Wahrscheinlichkeit, dass M der \textit{k-te} Wert in der zu sortierenden menge S ist, setzt sich aus drei möglichen Fällen zusammen:
      \begin{itemize}
      	\item Alle drei gezogenen Zahlen sind gleich.
      	\item Zwei der gezogenen Zahlen sind gleich.
      	\item Die drei gezogenen Zahlen sind verschieden.
      \end{itemize}
      Daraus ergibt sich:
      \begin{equation} \label{eq1}
      	\begin{split}
      	  P\{ M=s_k \} = & P\{ M=s_k , X_1 =X_2 =X_3 \} \\
      	  & + 3*P\{ M=s_k , X_1 < X_2 < X_3 \}\\
      	  & + 6*P\{ M=s_k , X_1 < X_2 < X_3 \}
      	  \end{split}
      \end{equation}
      \begin{equation} \label{eq2}
        \begin{split}
          P\{ M=s_k , X_1 =X_2 =X_3 \} & = P\{X_1 =s_k , X_2 =s_k , X_3 =s_k \}\\
          & = \frac{1}{n} * \frac{1}{n} * \frac{1}{n} = \frac{1}{n^3}
        \end{split}
      \end{equation}
      \begin{equation} \label{eq3}
      	\begin{split}
      	  P\{ M=s_k , X_1 \neq X_2 =X_3 \} & = P\{X_1 \neq s_k , X_2 =s_k , X_3 =s_k \}\\
      	  & = \frac{n-1}{n} * \frac{1}{n} * \frac{1}{n} = \frac{n-1}{n^3}
      	\end{split}
      \end{equation}
      \begin{equation} \label{eq4}
      	\begin{split}
      	  P\{ M=s_k , X_1 < X_2 < X_3 \} & = P\{X_1 < s_k , X_2 =s_k , X_3 < s_k \}\\
      	  & = \frac{k}{n} * \frac{1}{n} * \frac{n-1-k}{n} = \frac{k*(n-1-k)}{n^3}
      	\end{split}
      \end{equation}
      \newpage
      Wenn man nun die Ergebnisse der Gleichungen \ref{eq2}, \ref{eq3} und \ref{eq4} in die Gleichung \ref{eq1} einsetzt, erhält man folgendes Ergebnis:
      \begin{equation} \label{eq5}
      	\begin{split}
      	  P\{ M=s_k \} & = \frac{1}{n^3} + 3*\frac{n-1}{n^3} + 6*\frac{k*(n-1-k
      	  	)}{n^3} \\
      	  & = \frac{6*k*(n-1-k) + 3*n - 2}{n^3}
      	\end{split}
      \end{equation}
      
	
\end{document}