% !TEX TS-program = xelatex
%
\usepackage{lmodern}
\usepackage{amsmath}
\usepackage{amssymb}
\usepackage[ngerman]{babel}
\usepackage[procnames]{listings}
\usepackage{listings-rust}
\usepackage{listings}
\usepackage{ulem}
\usepackage{amsthm}
\usepackage{tikz}
\usepackage{wasysym}

\author{}

\lstloadlanguages{Matlab}
\lstloadlanguages{Ruby}

\lstset{%
  basicstyle = \ttfamily\color{black},
  commentstyle = \ttfamily\color{gray},
  keywordstyle = \ttfamily\color{red},
  stringstyle = \color{orange},
  breaklines = true,
  showstringspaces = false,
  procnamekeys={def},
  procnamestyle=\color{green}
}


\begin{document}
  \title{InformationsTheorie und Kryptologie: 4. Blatt für 12.4.2018}
  \maketitle

  \section*{10)}

  \definecolor{red}{HTML}{f92672}
  \definecolor{green}{HTML}{009900}


  \lstinputlisting[language=Ruby]{miller_rabin.rb}

  \subsection*{a)}

  \begin{align*}
    87 - 1 = 2^{1} \cdot 43
  \end{align*}

  Schlechte Werte für $x$: 1, 86

  Prozentsatz schlechter Werte: $2,3255813953488373 \% $

  \subsection*{b)}

  \begin{align*}
    89 - 1 = 2^{3} \cdot 11
  \end{align*}

  Schlechte Werte für $x$: 1, 2, 3, 4, 5, 6, 7, 8, 9, 10, 11, 12, 13, 14, 15, 16, 17, 18, 19, 20, 21, 22, 23, 24, 25, 26, 27, 28, 29, 30, 31, 32, 33, 34, 35, 36, 37, 38, 39, 40, 41, 42, 43, 44, 45, 46, 47, 48, 49, 50, 51, 52, 53, 54, 55, 56, 57, 58, 59, 60, 61, 62, 63, 64, 65, 66, 67, 68, 69, 70, 71, 72, 73, 74, 75, 76, 77, 78, 79, 80, 81, 82, 83, 84, 85, 86, 87, 88

  Prozentsatz schlechter Werte: $100.0 \%$


  \section*{11)}

    Quick-Sort in der \textit{Median-of-Three} Variante wählt zufällig 3 Werte aus der zu sortierenden Menge aus. Der Median dieser Werte wird als Pivot gewählt.\\

    \subsection*{a)}

      Die Wahrscheinlichkeit, dass M der \textit{k-te} Wert in der zu sortierenden menge S ist, setzt sich aus drei möglichen Fällen zusammen:
      \begin{itemize}
      	\item Alle drei gezogenen Zahlen sind gleich.
      	\item Zwei der gezogenen Zahlen sind gleich.
      	\item Die drei gezogenen Zahlen sind verschieden.
      \end{itemize}
      Daraus ergibt sich:
      \begin{equation} \label{eq1}
      	\begin{split}
      	  P\{ M=s_k \} = & P\{ M=s_k , X_1 =X_2 =X_3 \} \\
      	  & + 3*P\{ M=s_k , X_1 \neq X_2 = X_3 \}\\
      	  & + 6*P\{ M=s_k , X_1 < X_2 < X_3 \}
      	  \end{split}
      \end{equation}
      \begin{equation} \label{eq2}
        \begin{split}
          P\{ M=s_k , X_1 =X_2 =X_3 \} & = P\{X_1 =s_k , X_2 =s_k , X_3 =s_k \}\\
          & = \frac{1}{n} * \frac{1}{n} * \frac{1}{n} = \frac{1}{n^3}
        \end{split}
      \end{equation}
      \begin{equation} \label{eq3}
      	\begin{split}
      	  P\{ M=s_k , X_1 \neq X_2 =X_3 \} & = P\{X_1 \neq s_k , X_2 =s_k , X_3 =s_k \}\\
      	  & = \frac{n-1}{n} * \frac{1}{n} * \frac{1}{n} = \frac{n-1}{n^3}
      	\end{split}
      \end{equation}
      \begin{equation} \label{eq4}
      	\begin{split}
      	  P\{ M=s_k , X_1 < X_2 < X_3 \} & = P\{X_1 < s_k , X_2 =s_k , X_3 < s_k \}\\
      	  & = \frac{k}{n} * \frac{1}{n} * \frac{n-1-k}{n} = \frac{k*(n-1-k)}{n^3}
      	\end{split}
      \end{equation}
      \newpage
      Wenn man nun die Ergebnisse der Gleichungen \ref{eq2}, \ref{eq3} und \ref{eq4} in die Gleichung \ref{eq1} einsetzt, erhält man folgendes Ergebnis:
      \begin{equation} \label{eq5}
      	\begin{split}
      	  P\{ M=s_k \} & = \frac{1}{n^3} + 3*\frac{n-1}{n^3} + 6*\frac{k*(n-1-k
      	  	)}{n^3} \\
      	  & = \frac{6*k*(n-1-k) + 3*n - 2}{n^3}
      	\end{split}
      \end{equation}

    \subsection*{b)}
    \begin{tabular}{lr}
      \begin{tikzpicture}
        \draw[->] (-0.2,0) -- (5.5,0) node[right] {$k$};
        \draw[->] (0,-0.2) -- (0,4.2) node[above] {$P(k)$};
        \draw (0,0.3) to[bend left= 80] (5.5,0.3);
        \node at (-0.5, 1) {$\frac{1}{n}$};
        \node at (0.0, -0.5) {\scriptsize 0};
        \node at (5.5, -0.5) {\scriptsize $n - 1$};
        \node at (2.75, -1.2) {Median-of-Three Quick-Sort};
      \end{tikzpicture}
      &
      \begin{tikzpicture}
      	\draw[->] (-0.2,0) -- (5.5,0) node[right] {$k$};
      	\draw[->] (0,-0.2) -- (0,4.2) node[above] {$P(k)$};
        \draw[scale=1,domain=0:5.5,smooth,variable=\x,red] plot ({\x},{1});
        \node at (-0.5, 1) {$\frac{1}{n}$};
        \node at (0.0, -0.5) {\scriptsize 0};
        \node at (5.5, -0.5) {\scriptsize $n - 1$};
        \node at (2.75, -1.2) {Standard Quick-Sort};
      \end{tikzpicture}
    \end{tabular}

    \subsection*{c)}

      \textbf{Vorteile:}
      \begin{itemize}
      	\item Bessere Performance, da die Wahrscheinlichkeit ein optimales Pivot zu wählen viel größer ist als beim regulären \textit{Quick-Sort}.
      \end{itemize}
      \textbf{Nachteile:}
      \begin{itemize}
      	\item Extra Rechenaufwand bei der Bestimmung des Pivots.
      \end{itemize}

  \newpage
  \section*{12)}

    Wie groß ist die Wahrscheinlichkeit, dass unter 500 Menschen mindestens einer medial begabt ist. Ein Mensch gilt als medial begabt, wenn er beim Erraten der Ausgänge von 10 Münzwürfen maximal einen Fehler macht.\\
    \newline
    \textbf{Binomialverteilung:} $ P_{n,m} = \binom{n}{m} * p^m * (1-p)^{n-m} $ \\
    \newline
    Wahrscheinlichkeit mindestens 9 von 10 Münzwurfausgänge zu Erraten, wobei T einen erratenen Ausgang bezeichnet:
    \begin{equation}
      \begin{split}
        P\{ \# T \geq 9 \} & = \binom{10}{9} * (\frac{1}{2})^9 * (\frac{1}{2})^{10-9} \\
        & + \binom{10}{10} * (\frac{1}{2})^{10} * (\frac{1}{2})^{10-10} \\
        & = \binom{10}{9} * (\frac{1}{2})^{10} + \binom{10}{10} * (\frac{1}{2})^{10} \\
        & = 0.009765 \approx 0.98\% (\approx 9.77 \permil )
      \end{split}
    \end{equation}
    Wahrscheinlichkeit, dass keiner von 500 Menschen medial begabt ist:
    \begin{equation}
      \begin{split}
        P\{ \# M_{medial\_ begabt} = 0 \} & = \binom{500}{0} * (0.0009765)^{500} \\
        & = 0.007396 \approx 0.74\% (\approx 7.4 \permil )
      \end{split}
    \end{equation}
    Somit ist die Wahrscheinlichkeit, dass mindestens einer von 500 Menschen medial begabt ist:
    \begin{equation}
      \begin{split}
        P\{ \# M_{medial\_ begabt} > 0 \} & = 1 - P\{ \# M_{medial\_ begabt} = 0 \} \\
        & = 99.26\%
      \end{split}
    \end{equation}
    Daher ist mit \textbf{99.26}\% Wahrscheinlichkeit einer von 500 Personen medial begabt.

\end{document}
